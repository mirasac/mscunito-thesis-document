\section{Introduction}
\label{sec:Introduction}
Climate risk assessment is becoming central in contemporary activities related to the environment and in particular those whose assets could be affected by climate change.
In fact, an increasing number of organizations are incorporating climate change risk assessments into their decision processes.

A climate change risk assessment for a given system is the analysis of the impacts of and the responses to climate change regarding that system. Various guidelines are available for these kind of risk assessments (cf. \cite{2021ISO14091,2024EEAExecutiveSummary,2017GIZTheVulnerability,2014BowyerAdaptingTo}) and slight variations of them are adopted by authors, but they do not specify precisely the practical details of the assessment. In particular, there is no objective method to choose the climate indicators used in the assessment, but they are selected according to their effectiveness in scientific literature and in previous assessments, combined with the personal experience of the authors. The choice of the indicators is far from objective.



\subsection{Climate risk}
\label{sec:Climate risk}
Before introducing the methodology adopted to evaluate risk in the present work, first it is convenient to introduce a proper terminology.
In this work definitions by \gls{ISO} are used for their concision. When some terms are not available, they are taken from \gls{IPCC}. Both sources have similar definitions for the same terms.

\Gls{risk} is a general term which can be tailored to different contexts and applications as a measure of uncertain consequences on a system of interest.%
\footnote{Without delving into Philosophy, a source of change is needed to have consequences and it is specified by the definitions in use.}
A system is very broadly any concrete or abstract entity which can be affected by \gls{risk}.
\begin{example}
  Some possible systems which can be exposed to \glspl{risk} are any physical system, communities of people, an idea.
\end{example}
A paradigmatic example is the financial sector, where the concept of \gls{risk} is widely known and is connected directly to economic value and the concept of portfolio, to the point that financial risk management can be considered a research field itself.\cite{2004ChristoffersenElementsOf}
Examples on how other fields implement the concept are elaborated in \cite[14]{2017GIZRiskSupplement} and in \cite{2020ReisingerTheConcept}.

In this work the \gls{risk} related to climate change is: \glsdesc{risk}.%
\footnote{Note that this definition is not specific to climate \gls{risk} since no reference to climate is made.}
\Gls{IPCC} proposes a similar definition, expanding on the entities involved (e.g. the possible systems) and the contexts in which the term is used, but focusing only on negative effects: \glsuseri{risk}.
An important aspect of climate \gls{risk} is that it originates both from \gls{impact} of climate change, i.e. \glsdesc{impact}, and any response to it, i.e. \glsdesc{response}. It is not common to see \glspl{response} integrated into risk assessment, as exposed by \cite[492]{2021SimpsonAFramework}, and for the purposes of the present study they are neglected.
Henceforth, terms climate \gls{risk} and \gls{risk} are used interchangeably.

To make the assessment easily extensible and modular, \gls{risk} is defined as the result of the interaction of three elements, i.e. its \glspl{determinant}, namely \gls{hazard}, \gls{exposure} and \gls{vulnerability}. \Gls{response} is considered the fourth \gls{determinant} of \gls{risk}, when the adopted methodology includes it in the assessment.
Definitions of \glspl{determinant} were introduced in \cite[69-70]{2012FieldManagingThe} and offer a change of direction from previous methodologies centered on the concept of \gls{vulnerability} of the system instead of the overall \gls{risk} (cf. \cite{2017GIZTheVulnerability} and \cite{2017GIZRiskSupplement}).

The \gls{hazard} is defined by \gls{ISO} as \glsdesc{hazard} and is elaborated further by \gls{IPCC} on which subjects it applies to. No particular reference is made to the climate system in the definitions, hence \gls{IPCC} provides the more specific term \gls{CID} in \cite[2224]{2021MatthewsAnnexVII} to address to climate-related physical phenomena and with a neutral connotation (cf. \cite[10]{2020ReisingerTheConcept} or \cite[1871-1872]{2021RanasingheClimateChange}). In the following the term \gls{hazard} is used to address to climate-related \glspl{hazard} for brevity.

The \gls{exposure} of a system is determined by \glsdesc{exposure}.

The \gls{vulnerability} of a system is \glsdesc{vulnerability}. Properties of the system which determine its \gls{vulnerability} may be classified further in \gls{sensitivity}, i.e. \glsdesc{sensitivity}, and \gls{adaptive_capacity}, i.e. \glsdesc{adaptive_capacity}. This classification in general helps the analysis of the system and the identification of responses, e.g. \gls{adaptation} measures may increase the \gls{adaptive_capacity} of some elements of the system.

Each \gls{determinant} may be viewed as a collection of elements, which are of different nature depending on the \gls{determinant} they belong to, but are addressed generically as \glspl{driver}.%
\footnote{When this terminology is not applied, it is common to refer to \glspl{driver} with the name of the \glspl{determinant} they belong to, e.g. drivers within the \gls{vulnerability} determinant are simply called vulnerabilities.}
\Glspl{CID} with negative \glspl{impact} are effectively \glspl{driver} within the \gls{hazard} \gls{determinant} which are related to the climate system. \Gls{hazard} \glspl{driver} used in the present work are selected from the taxonomy provided by European Union for \gls{CCRA}, to have a well-known and authoritative reference in the field.\cite[177]{2024EU20212139}
Physical elements of the system may be effectively considered as \glspl{driver} of \gls{exposure}.%
\footnote{In literature different terms are used to refer to \glspl{driver} within the \gls{exposure}, e.g. \emph{assets} or \emph{exposed sample} in \cite{2022DeVivoRiskAssessment}.}
\begin{example}
  A tropical storm is a \gls{driver} within the \gls{hazard} \gls{determinant},\cite[15]{2017GIZRiskSupplement} income is a \gls{driver} within the \gls{vulnerability} \gls{determinant},\cite[493]{2021SimpsonAFramework} airport structures (e.g. runways, aprons, terminals) in an airport (i.e. the system) are \glspl{driver} within the \gls{exposure} \gls{determinant}.\cite[551]{2022DeVivoRiskAssessment} More examples are available in the references.
\end{example}
The concept of \gls{driver} of \gls{risk} is borrowed from \cite{2021SimpsonAFramework} to allow a smooth extension to methodologies where \gls{risk} is the result of complex interactions within and across \glspl{determinant}. In section~\ref{sec:Complex risk} this topic is described further.

For a quantitative \gls{CCRA}, numerical values must be associated to \glspl{driver}. These values are called \glspl{indicator} and defined by \gls{ISO} as: \glsdesc{indicator}.%
\footnote{The definition by \gls{IPCC} is not as general because focuses only on the climate system and there is no specific term for the same concept applied to the other \glspl{determinant}.}
There can be more than one way to describe numerically the same \gls{driver}, hence the choice is not unique and the resulting risk may be affected by it.
In the following, the term \gls{indicator} written alone refers to an \gls{indicator} of a \gls{driver} within the \gls{hazard} \gls{determinant}, to relax the lenghty wording. For the other \glspl{determinant} the full qualification is used.

Having introduced the definitions above, the various components of \gls{risk} can be arranged as in figure~\ref{fig:nomenclature}. It sums up the relation between the various components, highlightning the fact that risk depends on drivers from three independent categories and are quantified possibly in multiple ways.
\begin{figure}[h]
  \centering
  \includegraphics*[keepaspectratio=true,width=0.75\textwidth]{nomenclature}
  \caption{A possibile representation of the components of climate risk. Climate hazards can affect exposed and vulnerable elements of the system and determine a risk for it. Collectively these factors are called drivers and can grouped in the three independent determinants of risk: hazard, exposure and vulnerability. To provide quantitative results, each driver is described by numerical values, i.e. the indicators, each providing a different possible description or measure of the same driver.}
  \label{fig:nomenclature}
\end{figure}



\subsection{Methodology}
\label{sec:Methodology}
Restricting the treatise to climate-related applications is not sufficient to fix details on \gls{risk}, e.g. how to evaluate it from its \glspl{determinant}. These implementation details depend on which methodology is chosen to perform the risk assessment, which is presented in this section and is defined operatively in section~\ref{sec:Methods}.

The methodology of \gls{CCRA} applied in the present work follows \cite{2017GIZTheVulnerability} and its upgrade \cite{2017GIZRiskSupplement} to the concepts presented in section~\ref{sec:Climate risk}. The latter should be read in parallel with the former and superseds the outdated concepts.

This methodology is split in eight modules, each dependent on the previous ones. The following is an overview of them:
\begin{enumerate}
  \item understand the context in which the assessment is framed and identify objectives, scope and resources involved;\cite[39-53]{2017GIZTheVulnerability}
  \item \label{itm:module_2} identify \glspl{risk} and \glspl{impact} affecting the system under study and determine \glspl{driver} of \gls{hazard}, \gls{exposure} and \gls{vulnerability};\cite[26-41]{2017GIZRiskSupplement}
  \item \label{itm:module_3} choose \glspl{indicator} for each \gls{driver} of \gls{hazard}, \gls{exposure} and \gls{vulnerability};\cite[73-84]{2017GIZTheVulnerability}
  \item \label{itm:module_4} collect data and quantify \glspl{indicator};\cite[87-103]{2017GIZTheVulnerability}
  \item \label{itm:module_5} normalise \glspl{indicator} to allow their comparison;\cite[105-119]{2017GIZTheVulnerability}
  \item \label{itm:module_6} for each \gls{determinant}, weight normalised \glspl{indicator} and aggregate them into a single value;\cite[121-131]{2017GIZTheVulnerability}
  \item \label{itm:module_7} aggregate values for individual \glspl{determinant} into a single value for \gls{risk};\cite[133-141]{2017GIZTheVulnerability}
  \item \label{itm:module_8} present the results of the \gls{CCRA}.\cite[143-154]{2017GIZTheVulnerability}
\end{enumerate}
When the \gls{vulnerability} of the system is recalled, it is split into \gls{sensitivity} and \gls{adaptive_capacity} if possible.

Even if a complete application of this methodology does not fall into the purposes of the present study, each module is briefly addressed in section~\ref{sec:Case study} when case studies are treated.



\subsection{Complex risk}
\label{sec:Complex risk}
Unlike other types of risk assessment, e.g. probabilistic risk assessment, \gls{CCRA} focuses on interactions between \glspl{driver} instead of estimating their likelihood.\cite[20-21]{2017GIZRiskSupplement} However, it is common to study each \gls{determinant} in isolation. Instead, an integrated risk assessment which is able to relate \glspl{driver} within and across \glspl{determinant} would be able to describe the overall \gls{risk} more accurately.\cite[145-147]{2023IPCCClimateChange}

Interacting elements are not considered mainly because it is more difficult to formalise the interactions. In \cite{2021SimpsonAFramework} the list of complex \gls{risk} adopted by \gls{IPCC} in \gls{AR6} is extended, to allow more granularity in the assessment, and three categories of complex \gls{risk} are proposed. The objective is to build a framework which helps to address to complex interactions more easily, thus helping their adoption in \glspl{CCRA}. \Gls{response} is included in the \glspl{determinant}, e.g. to introduce negative effects on \gls{vulnerability} due to maladaptation.
Depending on what is the origin of \gls{risk}, the categories are:\cite[493]{2021SimpsonAFramework}
\begin{enumerate}
  \item \label{itm:category_1} interacting \glspl{driver} within the same \gls{determinant};
  \item \label{itm:category_2} interacting \glspl{driver} across different \glspl{determinant};
  \item \label{itm:category_3} interacting \glspl{risk}.
\end{enumerate}
In general, category~\ref{itm:category_1} is considered as long as the methodology admits an aggregation of \glspl{driver} to obtain each \gls{determinant}.
\begin{example}
  Flood \gls{risk} in a geographical area is assessed. First, only artificial constructions are considered as system.
  The change in time of precipitation and temperature, i.e. \glspl{driver} of \gls{risk} within the \gls{hazard} \gls{determinant}, are aggregated to give a measure of the \gls{hazard}. With this value and the analogous values of the other \glspl{determinant}, a category~\ref{itm:category_1} \gls{risk} can be evaluated, which measures the interaction of its \glspl{determinant} only defined by the methodology and lacking a particular meaning.
  
  A collateral change in soil properties due to precipitation and temperature is added to the study. This results in a decrease of soil \gls{adaptive_capacity}, hence an increase of the \gls{vulnerability} of buildings in that area. This interaction belongs to category~\ref{itm:category_2}.
  
  A second iteration of the \gls{CCRA} moves the focus to human activities in the area under study. When the corrispondent \gls{risk} is evaluated, it can be merged with the \gls{risk} value found previously to summarise the overall flood \gls{risk}, which becomes a category~\ref{itm:category_3} complex \gls{risk}.
\end{example}

All three categories of complex \gls{risk} are present in the methodology. In module~\ref{itm:module_6} \glspl{indicator} are aggregated within the same \gls{determinant} (i.e. category~\ref{itm:category_1} complex \gls{hazard}). Moreover, the concept of intermediate impact, which mediates \glspl{driver} of \gls{hazard} and \gls{vulnerability}, are a form of category~\ref{itm:category_2} complex \gls{hazard}.\cite[33]{2017GIZRiskSupplement} Sub-risks addressed in module~\ref{itm:module_7} may fall into category~\ref{itm:category_3}.

In the present work only category~\ref{itm:category_1} is considered, as there is no particular relation between \glspl{driver} within different \glspl{determinant}, except for the usual aggregation defined in module~\ref{itm:module_7} needed to obtain the final \gls{risk}. Nevertheless, extending the current study by introducing complexity throught the other categories may be interesting to test the robustness of the results.

From a mathematical point of view, complex interactions between \glspl{driver} translate to mathematical functions which relate \glspl{indicator}. They may be treated as additional \glspl{indicator} to consider in the aggregation of \glspl{determinant}.\cite[39-40]{2008OECDHandbookOn}
In the methodology used in this work, no particular interaction is considered between \glspl{driver} across \glspl{determinant}. \Glspl{indicator} are related only by the functions which are chosen in the aggregation procedures of modules~\ref{itm:module_6} and~\ref{itm:module_7}.



\subsection{Structure of the document}
Each section of the document treats a different aspect of the analysis. A general understanding of the concept of \gls{risk} and the associated terms are useful to frame the problem, they are presented in section~\ref{sec:Introduction}.

Given the relevance of data, the whole section~\ref{sec:Data} is dedicated to them. I information on climate datasets and system-dependent data are described, along with any elaboration applied. The problem and the methodology are described mathematically in section~\ref{sec:Methods}.

The methodology is then applied to two case studies in section~\ref{sec:Case study}, where system-specific data and results are analysed. In section~\ref{sec:Discussion} the application of the \gls{SA} to the case studies is assessed. Finally in section~\ref{sec:Conclusion} the final considerations on the study and its applicability are summarised.
