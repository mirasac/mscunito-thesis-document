\section{Conceptual overview}
\label{sec:Conceptual overview}
Climate change is shaping ecosystems and human activities and is becoming a central topic of discussion across all fields.\cite{2022PortnerClimateChange,2022PortnerTechnicalSummary} Effects of climate change have different severity, often changing with time and with the evolution of contexts they occur.\cite{2024EEAExecutiveSummary} These impacts may depend also on the spatial scale of the systems affected.\cite{2021Doblas-ReyesLinkingGlobal}

The knowledge of present and future impacts of climate change becomes essential to reduce adverse outcomes, as well as to leverage potential benefits. Even more relevant is the concept of risk which moves the focus from the impacts to their potential occurrence, broadening the possibilities for related studies and actions.
In fact, an increasing number of organisations are incorporating \glspl{CCRA} into their decision processes.\cite{2024Carlin2024Climate} A definition of \gls{risk_assessment} is given by \gls{IPCC} as \glsdesc{risk_assessment}, and \gls{CCRA} is its restriction to climate-related \glspl{risk}.%
\footnote{Terms \emph{climate change risk assessment} and \emph{climate risk assessment} are used equivalently in literature (cf. \cite[11]{2017GIZRiskSupplement} and \cite[20]{2017GIZTheVulnerability}, also \cite{2024LoyerInventoryOf}). In this document the former is used, to highlight the focus on risks arising from climate change.}
Other authorities employ similar definitions of \gls{risk_assessment} and remark its usefulness.\cite{2021ISO14091,2019UNDRRGlobalAssessment}

More generally, a \gls{risk_assessment} is essential part of the process of \gls{risk_management}, which uses the assessed information to reduce risk through the application of, e.g. policies, strategies, adaptation plans.\cite{2018ISO31000} The objectives of \gls{risk_management} are defined in the broad field of \gls{DRR}.\cite{2016UNSecretary-GeneralReportOf}

\Gls{CCRA} combine \gls{DRR} practices and concepts with climate information.\cite{2017GIZRiskSupplement,2012FieldManagingThe} Credible climate information are refined by scientific data and in cooperation with users and stakeholders. They are the product of \glspl{climate_service}, being the provision of climate-related data and information to assist decision-making.\cite{2012HewittTheGlobal}. An overview of the evolving field of \glspl{climate_service} is available in \cite[1431-1433]{2021Doblas-ReyesLinkingGlobal} and in \cite[1862-1869]{2021RanasingheClimateChange}.

Various regulations, standards and guidelines for \gls{CCRA} are available, but very few specify in detail the methodology to follow.\cite{2024LoyerInventoryOf} Methodologies differ by various means, e.g. the steps required to gather preliminar information on impacts and risks, the analysis of the system exposed, the evaluation of risk and its components, the presentation of outcomes (cf. \cite{2021ISO14091,2024EEAExecutiveSummary,2017GIZTheVulnerability,2014BowyerAdaptingTo,2024EU20212139}, also see \cite[10-11]{2024LoyerInventoryOf} and \cite[9]{2017GIZRiskSupplement}). This variety hinders the comparison of \gls{CCRA} outcomes, diminishing their credibility and interoperability.

A single methodology may be chosen to perform the \gls{CCRA}, but the issue presents once again since the methodology may not define operational details, e.g. functions and procedures involved. Implementation details are left to the authors of the \gls{CCRA}, who base the choice on their experience and on scientific literature, according to the principles of \glspl{climate_service}.
To evaluate the dependence of \gls{risk} on the functional representation of its components is the objective of this work.

The methodology of \gls{CCRA} applied in the present work follows \cite{2017GIZTheVulnerability} and its upgrade \cite{2017GIZRiskSupplement}. The latter should be read in parallel with the former and supersedes outdated concepts.



\subsection{Climate risk}
\label{sec:Climate risk}
Before stating the objective of the present study, first it is convenient to introduce a proper terminology.
Definitions by \gls{ISO} are used for their concision. When some terms are not available, they are taken from \gls{IPCC}. Both sources have similar definitions for the same terms.

\Gls{risk} is a general term which can be tailored to different contexts and applications as a measure of uncertain consequences on a system of interest.%
\footnote{Without delving into Philosophy, a source of change is needed to have consequences and it is specified by the definitions in use.}
A system is very broadly any concrete or abstract entity which can be affected by \gls{risk}.
\begin{example}
  Some possible systems which can be exposed to \glspl{risk} are any physical system, communities of people, an idea.
\end{example}
A paradigmatic example is the financial sector, where the concept of \gls{risk} is widely known and is connected directly to economic value and the concept of portfolio, to the point that financial risk management can be considered a research field itself.\cite{2004ChristoffersenElementsOf}
Examples on how other fields implement the concept are elaborated in \cite[14]{2017GIZRiskSupplement} and in \cite{2020ReisingerTheConcept}.

In this work the \gls{risk} related to climate change is: \glsdesc{risk}.%
\footnote{Note that this definition is not specific to climate \gls{risk} since no reference to climate is made.}
\Gls{IPCC} proposes a similar definition, expanding on the entities involved (e.g. the possible systems) and the contexts in which the term is used, but focusing only on negative effects: \glsuseri{risk}.
An important aspect of climate \gls{risk} is that it originates from both \gls{impact} of climate change, i.e. \glsdesc{impact}, and any response to it, i.e. \glsdesc{response}. It is not common to see \glspl{response} integrated into risk assessment, as exposed by \cite[492]{2021SimpsonAFramework}, and for the purposes of the present study they are neglected.
From this perspective, two types of climate \gls{risk} are studied in \glspl{CCRA}: transition risk and physical risk. The former regards \glspl{impact} on finance, economy and society caused by decarbonisation and transitioning to a sustainable economy, the latter is the risk related to the realisation of physical hazards. Both types of risk are assessed by organisations and financial institutions,\cite{2023CarlinThe2023,2023CarlinTechnicalSupplement} however only physical risks are considered in this study.
Henceforth, terms climate \gls{risk} and risk are used interchangeably and both refer to climate-related physical risks.

To make the assessment easily extensible and modular, \gls{risk} is defined as the result of the interaction of three elements, i.e. its \glspl{determinant}, namely \gls{hazard}, \gls{exposure} and \gls{vulnerability}. \Gls{response} is considered the fourth \gls{determinant} of \gls{risk}, when the adopted methodology includes it in the assessment.
Definitions of \glspl{determinant} were introduced in \cite[69-70]{2012FieldManagingThe} and offer a change of direction from previous methodologies centered on the concept of \gls{vulnerability} of the system instead of the overall \gls{risk} (cf. \cite{2017GIZTheVulnerability} and \cite{2017GIZRiskSupplement}).

The \gls{hazard} is defined by \gls{ISO} as \glsdesc{hazard} and is elaborated further by \gls{IPCC} on which subjects it applies to. No particular reference is made to the climate system in the definitions, hence \gls{IPCC} provides the more specific term \gls{CID} in \cite[2224]{2021MatthewsAnnexVII} to address to climate-related physical phenomena and with a neutral connotation (cf. \cite[10]{2020ReisingerTheConcept} or \cite[1871-1872]{2021RanasingheClimateChange}). In the following the term \gls{hazard} is used to address to climate-related \glspl{hazard} for brevity.

The \gls{exposure} of a system is determined by \glsdesc{exposure}.

The \gls{vulnerability} of a system is \glsdesc{vulnerability}. Properties of the system which determine its \gls{vulnerability} may be classified further in \gls{sensitivity}, i.e. \glsdesc{sensitivity}, and \gls{adaptive_capacity}, i.e. \glsdesc{adaptive_capacity}. This classification in general helps the analysis of the system and the identification of responses, e.g. \gls{adaptation} measures may increase the \gls{adaptive_capacity} of some elements of the system.

Each \gls{determinant} may be viewed as a collection of elements, which are of different nature depending on the \gls{determinant} they belong to, but are addressed generically as \glspl{driver}.%
\footnote{When this terminology is not applied, it is common to refer to \glspl{driver} with the name of the \glspl{determinant} they belong to, e.g. drivers within the \gls{vulnerability} determinant are simply called vulnerabilities.}
\Glspl{CID} with negative \glspl{impact} are effectively \glspl{driver} within the \gls{hazard} \gls{determinant} which are related to the climate system. \Gls{hazard} \glspl{driver} used in the present work are selected from the taxonomy provided by European Union for \gls{CCRA}, to have a well-known and authoritative reference in the field.\cite[177]{2024EU20212139}
Physical elements of the system may be effectively considered as \glspl{driver} of \gls{exposure}.%
\footnote{In literature different terms are used to refer to \glspl{driver} within the \gls{exposure}, e.g. \emph{assets} or \emph{exposed sample} in \cite{2022DeVivoRiskAssessment}.}
\begin{example}
  A tropical storm is a \gls{driver} within the \gls{hazard} \gls{determinant},\cite[15]{2017GIZRiskSupplement} income is a \gls{driver} within the \gls{vulnerability} \gls{determinant},\cite[493]{2021SimpsonAFramework} airport structures (e.g. runways, aprons, terminals) in an airport (i.e. the system) are \glspl{driver} within the \gls{exposure} \gls{determinant}.\cite[551]{2022DeVivoRiskAssessment} More examples are available in the references.
\end{example}
The concept of \gls{driver} of \gls{risk} is borrowed from \cite{2021SimpsonAFramework} to allow a smooth extension to methodologies where \gls{risk} is the result of complex interactions within and across \glspl{determinant}. In section~\ref{sec:Complex risk} this topic is described further.

For a quantitative \gls{CCRA}, numerical values must be associated to \glspl{driver}. These values are called \glspl{indicator} and defined by \gls{ISO} as: \glsdesc{indicator}.%
\footnote{The definition by \gls{IPCC} is not as general because focuses only on the climate system and there is no specific term for the same concept applied to the other \glspl{determinant}.}
There can be more than one way to describe numerically the same \gls{driver}, hence the choice is not unique and the resulting risk may be affected by it.
In the following, the term \gls{indicator} written alone refers to an \gls{indicator} of a \gls{driver} within the \gls{hazard} \gls{determinant}, to relax the lenghty wording. For the other \glspl{determinant} the full qualification is used.

Having introduced the definitions above, the various components of \gls{risk} can be arranged as in figure~\ref{fig:nomenclature}. It sums up the relation between the various components, highlightning the fact that risk depends on drivers from three independent categories and are quantified possibly in multiple ways.
\begin{figure}[h]
  \centering
  \includegraphics*[keepaspectratio=true,width=0.75\textwidth]{nomenclature}
  \caption{A possibile representation of the components of climate risk. Climate hazards can affect exposed and vulnerable elements of the system and determine a risk for it. Collectively these factors are called drivers and can grouped in the three independent determinants of risk: hazard, exposure and vulnerability. To provide quantitative results, each driver is described by numerical values, i.e. the indicators, each providing a different possible description or measure of the same driver.}
  \label{fig:nomenclature}
\end{figure}



\subsection{Complex risk}
\label{sec:Complex risk}
Unlike other types of risk assessment, e.g. probabilistic risk assessment, \gls{CCRA} focuses on interactions between \glspl{driver} instead of estimating their likelihood.\cite[20-21]{2017GIZRiskSupplement} However, it is common to study each \gls{determinant} in isolation. Instead, an integrated risk assessment which is able to relate \glspl{driver} within and across \glspl{determinant} would be able to describe the overall \gls{risk} more accurately.\cite[145-147]{2022BegumPointOf}

Interacting elements are not considered mainly because a proper formalisation of the interactions is difficult and rarely clear. In \cite{2021SimpsonAFramework} the list of complex \gls{risk} adopted by \gls{IPCC} in \gls{AR6} is extended, to allow more granularity in the assessment, and three categories of complex \gls{risk} are proposed. The objective is to build a framework which helps to address to complex interactions more easily, thus helping their adoption in \glspl{CCRA}. \Gls{response} is included in the \glspl{determinant}, e.g. to introduce negative effects on \gls{vulnerability} due to \gls{maladaptation}.
Depending on what is the origin of \gls{risk}, the categories are:\cite[493]{2021SimpsonAFramework}
\begin{enumerate}
  \item \label{itm:category_1} interacting \glspl{driver} within the same \gls{determinant};
  \item \label{itm:category_2} interacting \glspl{driver} across different \glspl{determinant};
  \item \label{itm:category_3} interacting \glspl{risk}.
\end{enumerate}
In general, category~\ref{itm:category_1} is considered as long as the methodology admits an aggregation of \glspl{driver} to obtain each \gls{determinant}.
\begin{example}
  Flood \gls{risk} in a geographical area is assessed. First, only artificial constructions are considered as system.
  The change in time of precipitation and temperature, i.e. \glspl{driver} of \gls{risk} within the \gls{hazard} \gls{determinant}, are aggregated to give a measure of the \gls{hazard}. With this value and the analogous values of the other \glspl{determinant}, a category~\ref{itm:category_1} \gls{risk} can be evaluated, which measures the interaction of its \glspl{determinant} only defined by the methodology and lacking a particular meaning.
  
  A collateral change in soil properties due to precipitation and temperature is added to the study. This results in a decrease of soil \gls{adaptive_capacity}, hence an increase of the \gls{vulnerability} of buildings in that area. This interaction belongs to category~\ref{itm:category_2}.
  
  A second iteration of the \gls{CCRA} moves the focus to human activities in the area under study. When the corrispondent \gls{risk} is evaluated, it can be merged with the \gls{risk} value found previously to summarise the overall flood \gls{risk}, which becomes a category~\ref{itm:category_3} complex \gls{risk}.
\end{example}

All three categories of complex \gls{risk} may be found in the methodology (see section~\ref{sec:Methodology of risk assessment} for details), but in the present work only category~\ref{itm:category_1} is considered, as there is no particular relation between \glspl{driver} within different \glspl{determinant}, except for the aggregation into the final \gls{risk} value.
Nevertheless, extending the current study by introducing complexity throught the other categories may be interesting to test the robustness of the results.

From a mathematical point of view, complex interactions between \glspl{driver} translate to mathematical functions which relate \glspl{indicator}. They may be treated as additional \glspl{indicator} to consider in the aggregation of \glspl{determinant}.\cite[39-40]{2008OECDHandbookOn}
In the chosen methodology, no specific interaction is considered between drivers across determinants. This translates into linear relations between indicators when they are aggregated.



\subsection{Problem statement}
The objective of this work is to show how the \gls{risk} evaluated following a given methodology depends on the choice of \glspl{indicator}. In particular, the study is restricted to indicators of \gls{hazard} and their definitions are modified by varying the parameters they depend on.

Technically this study resembles a \gls{SA}: the effects on the outcome of a system by varying its input factors are assessed and attributed to specific input factors.\cite[627-632]{2015DeanHandbookOf} In this case the system is the \gls{CCRA} and the methodology which implements it, its outcome is the \gls{risk} value and the input factors are the parameters.
More in detail, this work adopts a global approach to \gls{SA}, since the space of all possible parameters, or a significative subset of it, is explored. This is essential for a significative \gls{SA}, because \glspl{indicator} are generally non-linear functions of their parameters, hence the relation of the system on the input space is highly non linear.\cite[31-32]{2019SaltelliWhySo}

A \gls{SA} is normally preceded by an \gls{UA}, which quantifies the uncertainty on the output of the system by exploring the statistical properties of the system and its inputs.\cite[29-30]{2019SaltelliWhySo} The present study does not employ strictly an \gls{UA}, because probability distributions of parameters and their uncertainties are not considered explicitly, a non-parametric space-filling approach is adopted instead.
An estimation of uncertainties would be possible by employing multiple instances of the input space (e.g. bootstrapping, Monte Carlo methods). Additionally, uncertainties on climate data are considered only for projections, where a model ensemble is employed.

Results from this work highlight the importance of \glspl{indicator} parameters in a \gls{CCRA}. The methods may be useful to authors of \glspl{CCRA} to address the arbitrariness of parameters selection and to support the outcomes of the assessment. In this regard, the present study may found an application in \glspl{climate_service}.



\section{Structure of the document}
Each section of the document treats a different aspect of the analysis.
A general understanding of the concept of \gls{risk} and the associated terms are useful to frame the problem, they are presented in section~\ref{sec:Conceptual overview}.
Given the relevance of data, the whole section~\ref{sec:Data} is dedicated to them. Climate datasets and system-dependent data are described, along with any elaboration applied. The problem and the methodology of \gls{CCRA} are described mathematically in section~\ref{sec:Methodology of risk assessment}.
The methodology is then applied to two case studies in section~\ref{sec:Results}, where system-specific data and results are analysed and the actual \gls{SA} is performed.
In section~\ref{sec:Discussion} the effectiveness of the methods is assessed.
Finally in section~\ref{sec:Conclusion} the final considerations on the study and its applicability are summarised.
