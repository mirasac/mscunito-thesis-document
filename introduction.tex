Climate risk assessment is becoming central in contemporary activities related in any way to the environment and whose assets could be affected by climate change.
In particular, climate change risk assessment is a topic more and more organizations are considering in their decisions.

A climate change risk assessment for a given system is the analysis of the impacts of and the responses to climate change regarding that system. Various guidelines are available for these kind of risk assessments (cfr. \cite{2021ISO14091,2024EEAExecutiveSummary,2017GIZTheVulnerability}) and authors of assessments (e.g. consulting firms) follow a common procedure.
This procedure can be summarised in few steps:
\begin{enumerate}
  \item Collect requirements, documentation and information in general from clients and users about the system which is the subject of the assessment.
  \item Collect from the client data and information about climate, environment and exposed samples concerning the assessed system.
  \item Determine hazards potentially affecting the exposed samples of the system and their exposure and vulnerability.
  \item For each determinant, identify indicators suitable to describe the system.
  \item Quantify indicators using collected data.
  \item Unify previous information and climate projections to obtain the final risk, also projected into the future using specific climate change scenarios.
  \item Propose mitigation and adaptation measures and responses based on the outcome of the risk assessment.
\end{enumerate}

In general, slight variations of this procedure is adopted by authors and guidelines do not specify precisely the practical details of the assessment. In particular, there is no objective method to choose the climate indicators used in the assessment, but they are selected according to their effectiveness in scientific literature and in previous assessments, combined with the personal experience of the authors. The choice of the indicators is far from objective.
