\section{Data}
\label{sec:Data}
Climate data show great complexity in structure and availability (e.g. \glspl{ECV} can be represented as multidimensional objects, some climate datasets are collections of \glspl{ECV}). For these and other properties discussed in \cite{2014FaghmousABig}, climate data can be regarded as big data.

In this work a generic \gls{ECV} $V$ can be represented mathematically as a scalar function
\begin{equation}
  \label{eq:math_variable}
  V : S_\text{lat} \times S_\text{lon} \times S_\text{time} \to \mathbb{R}
\end{equation}
where $S_\text{lat}$, $S_\text{lon}$ and $S_\text{time}$ are domains of latitude, longitude and time dimensions, respectively.%
\footnote{In contests related to Machine Learning these objects are called tensors. Since they may not satisfy the mathematical definition of a tensor, in particular the map may not be multilinear and the numerical sets may not be vector spaces, no reference to such objects is made in this work.}
Every numerical value is equipped with proper units of measurement, to represent physical quantities correctly. As a consequence, the codomain in equation~\eqref{eq:math_variable} is partially wrong: with an abuse of notation, it represents only the magnitude of the \gls{ECV} and does not consider the unit of measurement. This is a small exception to simplify the notation and in the remainder of this document units of measurement are always addressed explicitly.

A more practical representation of $V$ is a multidimensional array, where values in the domain are coordinates associated to each dimension and each entry of the array is the result of $V$ evaluated on those coordinates. Figure~\ref{fig:multidimensional} shows this representation visually. In the following, this representation is used to simplify the discussion and same \gls{ECV} name is used both for the function and the multidimensional array.
\begin{figure}[h]
  \centering
  \includegraphics*[keepaspectratio=true,width=0.75\textwidth]{multidimensional}
  \caption{Representation of a generic \gls{ECV} as multidimensional array.}
  \label{fig:multidimensional}
\end{figure}

\begin{example}
  Near-Surface Air Temperature, symbol $\tas$, is available for some coordinates and timestamps. It can be seen as the scalar function in equation~\eqref{eq:math_variable}, which associates each value in set $S_\text{lat} \times S_\text{lon} \times S_\text{time}$ to a value with unit \unit{\kelvin}, or it can be represented as the multidimensional array in figure~\ref{fig:multidimensional}, where each entry is function~\eqref{eq:math_variable} evaluated at the corresponding coordinates.
\end{example}

\Glspl{normal} used as reference depends only on spatial coordinates, hence they are functions
\begin{equation}
  \label{eq:math_normal}
  \bar{V} : S_\text{lat} \times S_\text{lon} \to \mathbb{R}
\end{equation}
or equivalently, multidimensional arrays with spatial coordinates only.
\Glspl{normal} are evaluated as explained in \cite[6]{2017WorldMeteorologicalOrganizationWMOWMOGuidelines} using an averaging period specific to each case study (cf. section~\ref{sec:Case study}).

\Glspl{indicator} of \glspl{driver} within the \gls{hazard} \gls{determinant} are functions of \glspl{ECV} and additional parameters. In general they are aggregated over the temporal dimension and in this work their evaluation is performed for each year, i.e. the \gls{indicator} has yearly resolution or is evaluated with yearly frequency.
In this work an \gls{indicator} $I$ can be defined mathematically as
\begin{equation}
  \label{eq:math_indicator}
  I : S_\text{lat} \times S_\text{lon} \times S_\text{y} \times \prod_{p \in P_I} S_p \to \mathbb{R}
\end{equation}
where $S_\text{y}$ is a set of the years considered during the analysis, $P_I$ is the set of parameters for that indicator and $S_p$ is the set of values available for each parameter $p \in P_I$.%
\footnote{As a symbolic shortcut, if $P_I = \emptyset$ then the indicator is defined only over $S_\text{lat} \times S_\text{lon} \times S_\text{y}$.}
An \gls{indicator} can be represented as a multidimensional array, similarly to \glspl{ECV}.
The dependence of an \gls{indicator} on \glspl{ECV} is not clear in the definition given by equation~\eqref{eq:math_indicator}, but in the following this is made explicit by the context or by the definition of the \gls{indicator}.

\begin{example}
  The \gls{indicator} $\mathrm{TX_x}$ is evaluated for the period 1991-2020 with yearly frequency. This \gls{indicator} is the monthly maximum value of daily maximum temperature,\cite{ETCCDIClimate} hence:
  \begin{itemize}
    \item it depends on \gls{ECV} $\tasmax$ defined at daily frequency over the considered period,
      \begin{equation*}
        S_\text{time} = \left\{ t : \parbox{0.45\linewidth}{$t$ day from \DTMdisplaydate{1991}{1}{1}{-1} to \DTMdisplaydate{2020}{12}{31}{-1}} \right\}
        \quad ;
      \end{equation*}
    \item spatial dimensions are not specified, hence the evaluation is performed for each point of an arbitrary set $S_\text{lat} \times S_\text{lon}$;
    \item no additional parameters are required, $P_\mathrm{TX_x} = \emptyset$;
    \item the outcome is a scalar value for each year in the period,
      \begin{equation*}
        S_\text{y} = \{ 1991, 1992, \dots, 2020 \}
        \quad ;
      \end{equation*}
    \item the multidimensional array representation of the \gls{indicator} is in the following figure, where each entry is a real value in \unit{\kelvin}:
      \begin{center}
        \includegraphics*[keepaspectratio=true,width=0.50\linewidth]{indicator_climatological_standard_normal}
      \end{center}
  \end{itemize}
\end{example}
