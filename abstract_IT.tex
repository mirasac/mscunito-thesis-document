Gli impatti del cambiamento climatico hanno una vasta portata, interessando diversi settori e con diversa intensità. La valutazione del rischio da cambiamento climatico sta diventando uno strumento fondamentale per organizzazioni e stakeholder per supportare le loro decisioni e salvaguardare le loro attività. In generale, la valutazione del rischio e il concetto di rischio sono diffusi, nondimeno ogni applicazione ha caratteristiche proprie e la maggior parte delle volte l'interoperabilità è carente. Inoltre, i dati climatici e più in generale i servizi climatici forniscono informazioni scientifiche alla base di una valutazione del rischio da cambiamento climatico, generalmente veicolate tramite indicatori numerici. Tuttavia, la grande quantità di regolamenti, standard e linee guida non definisce una pratica consolidata e dettagliata per eseguire una valutazione e combinare le informazioni climatiche. Nel presente studio, viene adottata una metodologia di valutazione del rischio del cambiamento climatico per esplorare come particolari scelte di funzioni matematiche possano influenzare il risultato. Più in dettaglio, i parametri degli indicatori climatici sono fatti variare e la sensibilità del rischio a queste variazioni viene analizzata tramite grafici multidimensionali. Per quantificare il rischio, la valutazione del rischio da cambiamento climatico viene applicata all'aeroporto di Torino-Caselle, come caso di studio nel settore dell'aviazione. Vengono applicate varie procedure analitiche ai dati climatici per ottenere il valore finale di rischio. Statistiche multi-modello sono ottenute dal dataset disaggregato NEX-GDDP-CMIP6 basato sui dati ScenarioMIP di CMIP6, che è stato aggiustato dal bias utilizzando rianalisi del dataset ERA5. Le medie di ensemble multi-modello sono utilizzate per fornire stime del rischio in orizzonti temporali futuri. Tuttavia, l'incertezza domina alcuni risultati, pertanto diventa necessaria un'analisi dell'incertezza appropriata per caratterizzare gli errori derivati dagli input e la loro propagazione durante l'applicazione della metodologia. Ulteriori risultati confermano la variabilità del rischio nel tempo rispetto a un periodo di riferimento nel passato e attraverso diverse proiezioni di scenario del clima futuro, coerentemente con le loro definizioni. Questa variabilità è legata alla scelta dei parametri e crea un quadro interconnesso di valori dei parametri, periodi futuri scelti per la valutazione e indicatori climatici selezionati. Tutti questi elementi dovrebbero essere considerati insieme a un'analisi di sensibilità per fornire una robusta valutazione del rischio da cambiamento climatico.
