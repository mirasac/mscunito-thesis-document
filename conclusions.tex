The concept of risk is a powerful tool to manage undesired outcomes and cope with an uncertain future. Climate change is one of the main source of risk for the present and future of ecosystems and human activities. Acting ahead of time by understanding impacts of climate change and exploring viable responses to it is becoming fundamental.
This is the purpose of climate change risk assessment, which collect information from various stakeholders and involved parties to address to risks generated by climate change and which may affect any valuable system.

Operatively, a climate change risk assessment quantifies risk from climate change mixing climate data and available information on the system and at various time periods. How data are combined is relegated to the methodology in use and in the years many variants have been issued from different industries, organisations, political entities. This variability is a direct consequence of the extent of the concept of risk and related fields. One of the drawbacks of such richness of materials is the lack of a uniform approach to its application. This is transposed to climate-related fields and climate change risk assessment, where different methodologies suggest different implementations, without a real comparison between them or a robust procedure to translate the results.

In the present work a methodology of climate change risk assessment is chosen and applied to a system. The methodology includes the most recent interpretation of climate risk, intended as a function of hazard from climate change and vulnerability and exposure of the system. Each of them is a determinant of risk and their purpose is to describe various aspects of the risk studied and the system that can be impacted. Determinants are quantified functions called indicators, in particular the hazard indicators being functions of climate data and additional parameters.
The objective of the thesis is to study how variations of parameters values affect the outcome of the climate change risk assessment. An airport is chosen as system under study.

Climate data from the past are used to build a reference for discussing risk in the future. Reanalyses from ERA5 are used as reference to adjust the bias of projection dataset NEX-GDDP-CMIP6. Three different SSP scenarios from ScenarioMIP are used to increase the variability of the outcomes and the final risk values are the averages over an ensemble of 22 CMIP6 models.
Then the methodology of risk assessment is followed to identify system-dependent data and relevant indicators. The present work is restricted to two known sources of climate risk: heat wave and heavy precipitation. Indicators from literature are chosen and generalised to allow the variation of parameters they depend on. Both risks and indicators are selected to be meaningful for the application of the methodology to the case study.

Results of the study are risk values a functions of parameters in hazard indicators. Risk values are compared across time horizons and related to the risk of the reference period.

Plots generated for some indicators show the temporal evolution of risk and the differences across scenarios, which are coherent with their description of future climate change. Generally a climate change risk assessment as a limited spatial extension and the information related to the system are limited in time, depending on the context and scope of the assessment. Observations carried out in the first result suggest that parameters for hazard indicators should be adated to time horizons, to avoid excessive or meaningless risk values resulting from the assessment.

The multidimensional aspect of resulting data is leveraged to propose a procedure of selection of the number of consecutive days, which is a parameter in indicators of both temperature and precipitation. When the relevant indicators are included in the same climate change risk assessment, this procedure provides a rule to select univocally a parameter value, depending on the period of interest.

The present work tries to quantify some weaknesses of existent methodologies of climate change risk assessment with the support of computational tools. The procedures adopted here could be tested on different case studies and for more specific characterisations of climate risk, to provide and improve robust practical tools at each iteration.
A uniform but flexible framework for climate change risk assessment may seems unreachable, but as tiny steps as the present work may help to reduce the distance.
