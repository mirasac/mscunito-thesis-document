Impacts of climate change have wide reach, affecting different fields and with varied severity. Climate change risk assessment is becoming a fundamental tool for organisations and stakeholders to support their decisions and to safeguard their activities. In general, risk assessment and the concept of risk are widespread, however each application has its peculiarities and most of the times there is lack of interoperability. Additionally, climate data and more generally climate services provide scientific information at the base of a climate change risk assessment, generally conveyed through numerical indicators. However, the large amount of regulations, standards and guidelines does not define a well-established and detailed practice to perform an assessment and combine climate information.
In the present study, a methodology of climate change risk assessment is adopted to explore how particular choices of mathematical functions may affect the outcome. More in detail, parameters of climate indicators are varied and the sensitivity of risk to these variations is analysed through multidimensional plots.
To quantify risk, the climate change risk assessment is applied to the Torino-Caselle airport, as case study in the aviation sector. Various analytical procedures are applied to climate data to obtain the final risk value. Multi-model statistics are obtained from the downscaled dataset NEX-GDDP-CMIP6 based on ScenarioMIP data from CMIP6, which is bias adjusted using reanalyses from ERA5 dataset. Multi-model ensemble averages are used to provide estimates of risk in future time horizons. However, uncertainty dominates some of the results, therefore a proper uncertainty analysis becomes necessary to characterise errors derived from inputs and their propagation through the methodology.
Additional results confirm variability of risk in time with respect to a reference period in the past and across different scenario projections of future climate, coherently with their definitions. This variability is related to the choice of parameters and creates an interconnected picture of parameters values, future periods chosen for the assessment and selected climate indicators. All these elements should be considered in conjunction with a sensitivity analysis to provide a robust climate change risk assessment.
