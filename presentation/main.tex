%%%%%%%%% Packages %%%%%%%%%
\documentclass[aspectratio=43]{beamer}
\usepackage[T1]{fontenc}
\usepackage{pgfpages}

%%%%%%%%% Packages setup %%%%%%%%%
% \setbeameroption{show notes on second screen}
%\setbeameroption{show only notes}
%\setbeamerfont{note page}{size=\tiny}
\setbeamercolor{note page}{bg=white, fg=black}
\setbeamercolor{note title}{bg=white!99!black, fg=black}
%\usetheme{Hannover}
\usecolortheme{spruce}
\setbeamerfont{institute}{size=\scriptsize}

%%%%%%%%% Document informations %%%%%%%%%
\title{Sensitivity analysis of climate change risk assessment}
\subtitle{Study of parameters variation in hazard indicators}
\author[Marco Casari]{%
  Marco Casari\\[2ex]%
  {\scriptsize Supervisors}\\%
  Prof.ssa Elisa Palazzi\\%
  Prof. Alberto Viglione\\%
}
\date[04/07/2024]{Midterm discussion, 4 July 2024}
\institute[UniTo]{University of Turin}

%%%%%%%%% Document %%%%%%%%%
\begin{document}
\begin{frame}
  \titlepage
\end{frame}



\section{Introduction}
\begin{frame}{Introduction}
  \begin{itemize}
    \item Risk: potential for adverse consequences for human or ecological systems [...]
    % 2021MatthewsAnnexVII
    \note[item]{First few definitions to have a common starting point, all from IPCC AR6.}
    \item Climate Change Risk Assessment (CCRA)
    \note[item]{Estimation of risk related to climate change, i.e. determined by potential impacts of climate change and human responses to climate change.}
    % 2021MatthewsAnnexVII
    \note[item]{Some additional definitions are needed, e.g. determinants of risk, risk drivers.}
    % 2021SimpsonAFramework
    \note[item]{If quantitative, CCRA conveys numerical values combining the chosen indicators.}
  \end{itemize}
\end{frame}

\begin{frame}{The problem}
  \begin{itemize}
    \item<1-> The choice of indicators is arbitrary
    \note[item]<1->{Many methodologies and guidelines, different indicators may lead to different risks.}
    \item<2-> Sensitivity analysis of indicators within the hazard determinant
    \note[item]<1->{Many methodologies and guidelines.}
  \end{itemize}
\end{frame}

\begin{frame}{Case study}
  \begin{itemize}
    \item<1-> Torino Airport
    % MC picture of geographical location.
    \note[item]<1->{Exposure and vulnerability are fixed.}
    \item<2-> Hazard drivers: heat wave, heavy precipitation
    \note[item]<2->{Hazard drivers are from the European Taxonomy of climate hazards.}
    % 2024EU20212139
  \end{itemize}
\end{frame}



\section{Data}
\begin{frame}{Climate datasets}
  \begin{itemize}
    \item Climatological baseline: ERA5
    % 2018C3SERA5Hourly
    \item Climate projections: NEX-GDDP-CMIP6
    % NASAEarth
  \end{itemize}
\end{frame}

\begin{frame}{ERA5}
  \begin{itemize}
    \item Organisation: European Centre for Medium-Range Weather Forecasts
    \item Data type: reanalysis
    \item Spatial resolution: 0.25° x 0.25°
    \item Time frequency: hour
  \end{itemize}
\end{frame}

\begin{frame}{NEX-GDDP-CMIP6}
  \begin{itemize}
    \item Organisation: NASA Earth Exchange
    \item Data type: statistically downscaled bias-corrected climate projections
    \item Spatial resolution: 0.25° x 0.25°
    \item Time frequency: day
    \item Historical period 1950-2014, projection period 2015-2100
    \item Model: EC-Earth3
    \note[item]{Only model EC-Earth3 is considered for the midterm discussion.}
    \item Scenario: SSP1-2.6, SSP2-4.5, SSP5-8.5
  \end{itemize}
\end{frame}

\begin{frame}{Spatial frame}
  \begin{itemize}
    \item Box of 3 x 3 grid points centred at the coordinates of the airport
    \note[item]{}
  \end{itemize}
\end{frame}

\begin{frame}{Temporal frame}
  \begin{itemize}
    \item Baseline period: 1994-2023
    \item Time horizons: 2021-2040, 2051-2070, 2081-2100
  \end{itemize}
\end{frame}



\section{Methods}
% MC show with schemes and matrices the idea of varying the hazard indicators and study the statistics of the resulting ensembles.

\begin{frame}{Methodology}
  \begin{itemize}
    \item<1-> Indicators: heat wave frequency, maximum $n$-days precipitation
    \note[item]{The $n$ is one of the parameters of the indicator. Select intervals of variation of parameters and evaluate indicators for each combination of them.}
    \item<2-> Fix exposure and vulnerability from literature
    \item<3-> Evaluate risk
    %\note[item]{Risk is evaluated as average from }
  \end{itemize}
\end{frame}

\begin{frame}{Preprocessing}
  \begin{enumerate}
    \item<1-> Regrid ERA5
    \note[item]<1->{Since resolution is the same, a simple traslation of coordinates is sufficient. Use convention of NEX-GDDP-CMIP6: coordinates are the centre of the grid points.}
    \item<2-> Aggregate ERA5 at daily frequency
    \note[item]<2->{Total precipitation is summed, other quantities are averaged.}
    \item<3-> Align NEX-GDDP-CMIP6 timestamps
    \note[item]<3->{At daily resolution timestamps differ by hours of the same day, the offset loses meaning when aggregation at daily frequency is applied. Use convention of ERA5: timestamps point to the start of the day.}
    \item<4-> Bias adjustment
    \note[item]<4->{Training period 1950-1999, test period 2000-2014. Quantile Delta Mapping for both temperature and precipitation, show p-p plot for qualitative assessment of the adjustment.}
    % MC extend historical period to 2023 with one period from the projections?
  \end{enumerate}
\end{frame}

\begin{frame}{Evaluation of hazard indicators}
  \begin{enumerate}
    \item<1-> Define intervals of parameters
    \note[item]<1->{Less dense for regions of slowly varying quantile function.}
    \item<2-> Spatial aggregation
    \note[item]<2->{Sample average, check standard deviation.}
    % MC try EOF?
    \item<3-> Temporal aggregation
    \note[item]<3->{Sample average, check standard deviation.}
    % MC try Mann-Kendall, check compatiblity of Senn slope with 0?
    \item<4-> Risk evaluation
    \note[item]<4->{Given exposure and vulnerability, evaluate the final risk.}
  \end{enumerate}
\end{frame}

%\section{Case study}
% MC talk about airports, the hazard drivers affecting them and the chosen indicators.

\begin{frame}{Next steps}
  \begin{itemize}
    \item Uncertainty evaluation
    \item Evaluate risk with non-linear relations among hazard indicators and among determinants
    \item Extend analysis to Bologna's and Ciampino's airports
  \end{itemize}
\end{frame}

%\section{Results and discussion}
% MC show some results and their implications.

%\section{Conclusion}
% MC resume the initial idea and how the results obtained at this point match it.
% MC list some criticalities I am having at this point.

% MC make a subsection on open points.

\end{document}
