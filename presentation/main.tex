%%%%%%%%% Packages %%%%%%%%%
\documentclass[aspectratio=43]{beamer}
\usepackage[T1]{fontenc}
\usepackage{pgfpages}

%%%%%%%%% Packages setup %%%%%%%%%
\setbeameroption{show notes on second screen}
%\setbeameroption{show only notes}
%\setbeamerfont{note page}{size=\tiny}
\setbeamercolor{note page}{bg=white, fg=black}
\setbeamercolor{note title}{bg=white!99!black, fg=black}
\usetheme{Hannover}
\usecolortheme{spruce}

%%%%%%%%% Document informations %%%%%%%%%
\title{NA}  % MC find a title.
\author{Marco Casari}
\date[04/05/2024]{Midterm thesis exam, 4 July 2024}
\institute[UniTo]{University of Turin}

%%%%%%%%% Document %%%%%%%%%
\begin{document}

\section{Introduction}
% MC state the problem.
% MC sum up the proposed resolution.
% MC sho a slide about nomenclature.

\section{Climate datasets}
% MC list the used datasets, distinguish between historical period and climate projections and explain the choice (e.g. for climate projections describe why some scenarios are chosen and which periods).

\section{Methods}
% MC list the technology used for the analysis (e.g. Python, xarray) in a slide that will be just flashed.
% MC show with schemes and matrices the idea of varying the hazard indicators and study the statistics of the resulting ensembles.
% MC do not show code.

\section{Case study}
% MC talk about airports, the hazard drivers affecting them and the chosen indicators.

\section{Results and discussion}
% MC show some results and their implications.

\section{Conclusion}
% MC resume the initial idea and how the results obtained at this point match it.
% MC list some criticalities I am having at this point.

\end{document}
