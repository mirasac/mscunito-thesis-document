\newglossaryentry{determinant}{
  name={determinant},
  description={any component of risk, i.e. hazard, exposure, vulnerability, response, from \cite[493]{2021SimpsonAFramework}}
}
\newglossaryentry{risk}{
  name={risk},
  description={\emph{effect of uncertainty}, from \cite{2021ISO14091}},
  user1={\emph{The potential for adverse consequences for human or ecological systems, recognizing the diversity of values and objectives associated with such systems. In the context of climate change, risks can arise from potential impacts of climate change as well as human responses to climate change. Relevant adverse consequences include those on lives, livelihoods, health and well-being, economic, social and cultural assets and investments, infrastructure, services (including ecosystem services), ecosystems and species. [...]}, from \cite[2246]{2021MatthewsAnnexVII}}
}
\newglossaryentry{impact}{
  name={impact},
  description={\emph{effect on natural and human systems (3.3)}, from \cite{2021ISO14091}},
  user1={\emph{The consequences of realized risks on natural and human systems, where risks result from the interactions of climate-related hazards (including extreme weather/climate events), exposure, and vulnerability. Impacts generally refer to effects on lives, livelihoods, health and well-being, ecosystems and species, economic, social and cultural assets, services (including ecosystem services), and infrastructure. Impacts may be referred to as consequences or outcomes and can be adverse or beneficial.}, from \cite[2235]{2021MatthewsAnnexVII}}
}
\newglossaryentry{response}{
  name={response},
  description={action enact to mitigate the effects of climate change or adapt to it}
}
\newglossaryentry{mitigation}{
  name={mitigation},
  description={\emph{A human intervention to reduce emissions or enhance the sinks of greenhouse gases.}, from \cite[2239]{2021MatthewsAnnexVII}}
}
\newglossaryentry{adaptation}{
  name={adaptation},
  description={\emph{process of adjustment to actual or expected climate (3.4) and its effects}, from \cite{2021ISO14091}},
  user1={\emph{In human systems, the process of adjustment to
  actual or expected climate and its effects, in order to moderate harm or exploit beneficial opportunities. In natural systems, the process of adjustment to actual climate and its effects; human intervention may facilitate adjustment to expected climate and its effects.}, from \cite[2216]{2021MatthewsAnnexVII}}
}
\newglossaryentry{hazard}{
  name={hazard},
  description={\emph{potential source of harm}, from \cite{2021ISO14091}},
  user1={\emph{The potential occurrence of a natural or human-induced physical event or trend that may cause loss of life, injury, or other health impacts, as well as damage and loss to property, infrastructure, livelihoods, service provision, ecosystems and environmental resources}, from \cite[2233]{2021MatthewsAnnexVII}}
}
\newglossaryentry{driver}{
  name={driver},
  description={individual component of a determinant, from \cite[493]{2021SimpsonAFramework}},
  see={determinant,hazard,vulnerability,exposure,response}
}
\newglossaryentry{indicator}{
  name={indicator},
  description={\emph{quantitative, qualitative or binary variable that can be measured or described, in response to a defined criterion}, from \cite{2021ISO14091}},
  user1={\emph{Measures of the climate system, including large-scale variables and climate proxies.}, from \cite[2222]{2021MatthewsAnnexVII}}
}
\newglossaryentry{organization}{
  name={organization},
  description={\emph{person or group of people that has its own functions with responsibilities, authorities and relationships to achieve its objectives}, from \cite{2021ISO14091}}
}
\newglossaryentry{climate_index}{
  name={climate index},
  description={\emph{A time series constructed from climate variables that provides an aggregate summary of the state of the climate system}, cfr. \cite[2222]{2021MatthewsAnnexVII}}
}
\newglossaryentry{vulnerability}{
  name={vulnerability},
  description={\emph{propensity or predisposition to be adversely affected}, from \cite{2021ISO14091}},
  user1={\emph{The propensity or predisposition to be adversely affected. Vulnerability encompasses a variety of concepts and elements including sensitivity or susceptibility to harm and lack of capacity to cope and adapt.}, from \cite[2253]{2021MatthewsAnnexVII}},
  see={sensitivity,adaptive_capacity}
}
\newglossaryentry{sensitivity}{
  name={sensitivity},
  description={\emph{degree to which a system (3.3) or species is affected, either adversely or beneficially, by climate (3.4) variability or change}, from \cite{2021ISO14091}}
}
\newglossaryentry{adaptive_capacity}{
  name={adaptive capacity},
  description={\emph{ability of systems (3.3), institutions, humans, and other organisms to adjust to potential damage, to take advantage of opportunities, or to respond to consequences}, from \cite{2021ISO14091}}
}
\newglossaryentry{exposure}{
  name={exposure},
  description={\emph{presence of people, livelihoods, species or ecosystems, environmental functions, services, resources, infrastructure, or economic, social or cultural assets in places and settings that could be affected}, from \cite{2021ISO14091}},
  user1={\emph{The presence of people; livelihoods; species or ecosystems; environmental functions, services, and resources; infrastructure; or economic, social, or cultural assets in places and settings that could be adversely affected.}, from \cite[2229]{2021MatthewsAnnexVII}}
}
\newglossaryentry{normal}{
  name={normal},
  description={\emph{period averages computed for a uniform and relatively long period comprising at least three consecutive ten-year periods}, from \cite[2]{2017WorldMeteorologicalOrganizationWMOWMOGuidelines}},
  see={period_average}
}
\newglossaryentry{period_average}{
  name={period average},
  description={\emph{averages of climatological data computed for any period of at least ten years starting on 1 January of a year ending with the digit 1}, from \cite[2]{2017WorldMeteorologicalOrganizationWMOWMOGuidelines}},
  see={average}
}
\newglossaryentry{climatological_standard_normal}{
  name={climatological standard normal},
  description={\emph{averages of climatological data computed for the following consecutive periods of 30 years: 1 January 1981-31 December 2010, 1 January 1991-31 December 2020, and so forth}, from \cite[2]{2017WorldMeteorologicalOrganizationWMOWMOGuidelines}},
  see={average}
}
\newglossaryentry{average}{
  name={average},
  description={\emph{the mean of monthly values of climatological data over any specified period of time (not necessarily starting in a year ending with the digit 1). In some sources, this is also referred to as \emph{provisional normal}}, from \cite[2]{2017WorldMeteorologicalOrganizationWMOWMOGuidelines}}
}
\newglossaryentry{anomaly}{
  name={anomaly},
  plural={anomalies},
  description={\emph{The deviation of a variable from its value averaged over a reference period.}, from \cite[2218]{2021MatthewsAnnexVII}}
}
\newglossaryentry{data_assimilation}{
  name={data assimilation},
  description={\emph{Mathematical method used to combine different sources of information in order to produce the best possible estimate of the state of a system. This information usually consists of observations of the system and a numerical model of the system evolution. Data assimilation techniques are used to create initial conditions for weather forecast models and to construct reanalyses describing the trajectory of the climate system over the time period covered by the observations.}, from \cite[2225]{2021MatthewsAnnexVII}},
  see={reanalysis}
}
\newglossaryentry{reanalysis}{
  name={reanalysis},
  plural={reanalyses},
  description={\emph{Reanalyses are created by processing past meteorological or oceanographic data using fixed state-of-the-art weather forecasting or ocean circulation models with data assimilation techniques. They are used to provide estimates of variables such as historical atmospheric temperature and wind or oceanographic temperature and currents, and other quantities. Using fixed data assimilation avoids effects from the changing analysis system that occur in operational analyses. Although continuity is improved, global reanalyses still suffer from changing coverage and biases in the observing systems.}, from \cite[2245]{2021MatthewsAnnexVII}},
  see={data_assimilation}
}
