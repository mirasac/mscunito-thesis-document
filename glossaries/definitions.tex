\newglossaryentry{determinant}{
  name={determinant},
  description={Any component of risk, i.e. hazard, exposure, vulnerability, response, from \cite[493]{2023SimpsonAdaptationTo}.}
}
\newglossaryentry{hazard}{
  name={hazard},
  description={Here is used the IPCC's definition: ``The potential occurrence of a natural or human-induced physical event or trend that may cause loss of life, injury, or other health impacts, as well as damage and loss to property, infrastructure, livelihoods, service provision, ecosystems and environmental resources'', from \cite[2233]{2021MatthewsAnnexVII}. Note that this concept focues on the negative impacts of the physical drivers, the term \emph{Climatic impact-driver} is used for conditions with more general impacts (cfr. \cite[2224]{2021MatthewsAnnexVII}, \cite[10]{2020ReisingerGuidanceFor} and \cite[1871]{2021RanasingheClimateChange}). In this document the term Driver is used.},
  see={driver}
}
\newglossaryentry{driver}{
  name={driver},
  description={Individual components of determinants, from \cite[493]{2023SimpsonAdaptationTo}.}
}
\newglossaryentry{exposed sample}{
  name={exposed sample},
  description={asset that is threatened, e.g. airport runways, from \cite[553]{2022DeVivoRiskAssessment}.}
}
\newglossaryentry{indicator}{
  name={indicator},
  description={``quantitative, qualitative or binary variable that can be measured or described, in response to a defined criterion'', from \cite{2021ISO14091} and also used in \cite{2022DeVivoRiskAssessment,2023DeVivoApplicationOf,2023DeVivoClimate-RiskAssessment}. Some sources use the term \emph{metric}.}
}
\newglossaryentry{organization}{
  name={organization},
  description={``person or group of people that has its own functions with responsibilities, authorities and relationships to achieve its objectives'', from \cite{2021ISO14091}.}
}
\newglossaryentry{climate index}{
  name={cliamte index},
  description={``A time series constructed from climate variables that provides an aggregate summary of the state of the climate system'', cfr. \cite{2021MatthewsAnnexVII}.}
}
\newglossaryentry{vulnerability}{
  name={vulnerability},
  description={some sources evaluate the vulnerability determinant as a function of sensitivity and adaptive capacity,\cite[559]{2022DeVivoRiskAssessment} others also include the exposure determinant.\cite[58]{2017GIZTheVulnerability} In this document the first meaning is adopted, in particular as linear combination of sensitivity and adaptive capacity.}
}
\newglossaryentry{normal}{
  name={normal},
  description={``period averages computed for a uniform and relatively long period comprising at least three consecutive ten-year periods'', from \cite[2]{2017WorldMeteorologicalOrganizationWMOWMOGuidelines}.},
  see={period_average}
}
\newglossaryentry{period_average}{
  name={period average},
  description={``averages of climatological data computed for any period of at least ten years starting on 1 January of a year ending with the digit 1'', from \cite[2]{2017WorldMeteorologicalOrganizationWMOWMOGuidelines}.},
  see={average}
}
\newglossaryentry{average}{
  name={average},
  description={``the mean of monthly values of climatological data over any specified period of time (not necessarily starting in a year ending with the digit 1). In some sources, this is also referred to as ``provisional normal'''', from \cite[2]{2017WorldMeteorologicalOrganizationWMOWMOGuidelines}.}
}
