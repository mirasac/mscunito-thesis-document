The landview of terms and definitions used in climate change risk assessment is varied and this may cause confusion. For the sake of clarity, some definitions are listed below, along with the sources they are taken from.

If no specification of the source is present, the definition is assumed to be taken from \cite{2021ISO14091} or \cite{2021MatthewsAnnexVII}. Terms which are present in both sources have equivalent definitions.

\begin{description}
  \item[determinant] Any component of risk, i.e. hazard, exposure, vulnerability, response, from \cite[493]{2023SimpsonAdaptationTo}.
  \item[hazard] Here is used the IPCC's definition: ``The potential occurrence of a natural or human-induced
  physical event or trend that may cause loss of life, injury, or other
  health impacts, as well as damage and loss to property, infrastructure,
  livelihoods, service provision, ecosystems and environmental
  resources'', from \cite[2233]{2021MatthewsAnnexVII}. Note that this concept focues on the negative impacts of the physical drivers, the term \emph{Climatic impact-driver} is used for conditions with more general impacts (cfr. \cite[2224]{2021MatthewsAnnexVII}, \cite[10]{2020ReisingerGuidanceFor} and \cite[1871]{2021RanasingheClimateChange}). In this document the term Driver is used (cfr. definition~\ref{def:driver}).
  \label{def:driver}\item[driver] Individual components of determinants, from \cite[493]{2023SimpsonAdaptationTo}.
  \item[exposed sample] asset that is threatened, e.g. airport runways, from \cite[553]{2022DeVivoRiskAssessment}.
  \item[indicator] ``quantitative, qualitative or binary variable that can be measured or described, in response to a defined criterion'', from \cite{2021ISO14091} and also used in \cite{2022DeVivoRiskAssessment,2023DeVivoApplicationOf,2023DeVivoClimate-RiskAssessment}. Some sources use the term \emph{metric}.
  \item[organization] ``person or group of people that has its own functions with responsibilities, authorities and relationships to achieve its objectives'', from \cite{2021ISO14091}.
  \item[climate index] ``A time series constructed from climate variables that provides an aggregate summary of the state of the climate system'', cfr. \cite{2021MatthewsAnnexVII}.
  \item[vulnerability] some sources evaluate the vulnerability determinant as a function of sensitivity and adaptive capacity,\cite[559]{2022DeVivoRiskAssessment} others also include the exposure determinant.\cite[58]{2017GIZTheVulnerability} In this document the first meaning is adopted, in particular as linear combination of sensitivity and adaptive capacity.
\end{description}
