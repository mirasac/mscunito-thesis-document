The landview of terms and definitions used in climate change risk assessment is varied and this may cause confusion. For the sake of clarity, some definitions are listed below, along with the sources they are taken from.

If no specification of the source is present, the definition is assumed to be taken from \cite{2021ISO14091} or \cite{2021MatthewsAnnexVII}. Terms which are present in both sources have equivalent definitions.

\begin{description}
  \item[determinant] Any component of risk, i.e. hazard, exposure, vulnerability, response, from \cite[493]{2023SimpsonAdaptationTo}.
  \item[driver] Individual components of determinants, from \cite[493]{2023SimpsonAdaptationTo}.
  \item[exposed sample] asset that is threatened, e.g. airport runways, from \cite[553]{2022DeVivoRiskAssessment}.
  \item[indicator] ``quantitative, qualitative or binary variable that can be measured or described, in response to a defined criterion'', from \cite{2021ISO14091} and used also in \cite{2022DeVivoRiskAssessment,2023DeVivoApplicationOf,2023DeVivoClimate-RiskAssessment}.
  \item[organization] ``person or group of people that has its own functions with responsibilities, authorities and relationships to achieve its objectives'', from \cite{2021ISO14091}.
  \item[climate index] ``A time series constructed from climate variables that provides an aggregate summary of the state of the climate system'', cfr. \cite{2021MatthewsAnnexVII}.
\end{description}
