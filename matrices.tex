\documentclass{article}
\usepackage{amsmath}
\usepackage{amssymb}

\begin{document}
An indicator (i.e. a climate indicator for drivers within the hazard determinant) is a function
\begin{equation}
  \label{eq:indicator}
  h : \mathcal{C} \to \mathbb{R}
\end{equation}
where $\mathcal{C}$ is the set of climate variables. When more hazard indicators are considered, a subscript is added to identify each of them (e.g. the symbol $h_j$ with $j = 1, 2, 3$ is the shorthand for functions $h_1$, $h_2$ and $h_3$).

Once the system is identified and analysed from a structural point of view, the set of assets composing it is available. This set is represented as
\begin{equation}
  \mathcal{A} = \{ a_i : i = 1, \dots, n_A \}
\end{equation}
with $n_A$ number of assets, which is constant throughout the project.

The set of chosen hazard indicators is
\begin{equation}
  \mathcal{H} = \{ h_i : i = 1, \dots, n_H \}
\end{equation}
with $n_H$ number of hazard drivers from the European Taxonomy.  % MC cite the EU taxonomy.
Each element is a function of climate variables
\begin{equation}
  \forall i \in \{ 1, \dots, n_H \} \quad h_i : \mathcal{C} \to \mathbb{R}
\end{equation}
where $\mathcal{C}$ is the set of climate variables.

The set of indicators built by the interaction from existent hazard drivers is
\begin{equation}
  \{ h_{n_H + i} : i = 1, \dots, n_{X_j} \}
\end{equation}
with $n_{X_j}$ number of additional hazard drivers considered in the $j$-th experiment. This number might change in each experiment for category 2 complex risks.

Considering experiments for category 1 complex risks, the resulting \emph{hazard matrix}, i.e. the matrix containing values of hazard indicators for each asset, is $H \in \mathbb{R}^{n_A \times n_H}$
\begin{equation}
  H =
  \begin{bmatrix}
    h_{1, 1}   & \cdots & h_{1, n_H}   \\
    \vdots     & \ddots & \vdots       \\
    h_{n_A, 1} & \cdots & h_{n_A, n_H}
  \end{bmatrix}
\end{equation}
Once a particular choice of indicators is made, values of this matrix change when different datasets are considered.

When category 2 complex risks are considered, the hazard matrix is extended adding new columns corresponding to indicators which model complex interactions between hazard drivers. The new indicators are collected in a matrix $H_{X_j} \in \mathbb{R}^{n_A \times n_{X_j}}$.
Using block notation the extended matrix becomes $[H | H_{X_j}]$ for the $j$-th experiment and all the analysis performed on category 1 complex risks are extended directly to this matrix since they operate only on the rows.

\end{document}
