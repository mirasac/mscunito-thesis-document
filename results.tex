\section{Case study}
\label{sec:Case study}
This section presents briefly scope and objectives of the \gls{CCRA} as indicated by module~\ref{itm:module_1} of the methodology. There is no direct involvement of experts and stakeholders, but literature provides for information on context and objectives for the case study.

The present work is applied to impacts of climate change on airports.
The aviation sector is a complete test case for responses to climate change: it implements reduction of negative \glspl{impact} on infrastructures and transport network through adaptation measures as well as mitigation of the adverse effects on climate originated from sectorial activities \cite{2022ICAOICAOEnvironmental}. The disruption of critical infrastructures such as airports have important consequences on mobility and economic growth (see \cite{2018ICAOClimateAdaptation}, \cite[15]{2016BurbidgeAdaptingEuropean} and \cite[548]{2022DeVivoRiskAssessment} for a review of impacts). Hence, the estimation of climate \gls{risk} for airports becomes an essential tool for effective planning and risk management.

\Glspl{impact} from extreme temperatures and precipitation are studied, as they are among the biggest challenges to address in the aviation sector. Both \Gls{ICAO} and \gls{WMO} collected opinions and experiences from stakeholders through surveys \cite[62]{2018ICAOClimateAdaptation} and \cite[34]{2020WorldMeteorologicalOrganizationWMOOutcomesOf}, respectively.
More in detail, the two \gls{hazard} \glspl{driver} \gls{heat_wave} and \gls{heavy_precipitation} are considered. They are selected from the taxonomy provided by European Union for \gls{CCRA}, to have a well-known and authoritative reference in the field \cite[177]{2024EU20212139}. Both drivers are associated to acute physical risks.

An \gls{heat_wave} is defined by \gls{IPCC} as \glsdesc{heat_wave}. \Gls{WMO} presents a similar definition, highlighting the lack of a universally accepted definition, which is replaced by local criteria, although it would facilitate the exchange of information between actors \cite[5]{2023WorldMeteorologicalOrganizationWMOGuidelinesOn}. Heat waves have a wide range of impacts, particularly at the regional scale for airports, e.g. higher temperatures may damage runaways or decrease the efficiency of cooling systems, changes in air density affect the runaway length required for airside operations, increased humidity translates in more fog in earlier hours of the day \cite[23-28]{2018ICAOClimateAdaptation}.

From \gls{IPCC} \glsdesc{heavy_precipitation}, where the term \emph{precipitation} refers to water in every state, e.g. rain, snow, ice. Further details on the characterisation of heavy precipitation are provided by \gls{WMO} in \cite[6-7]{2023WorldMeteorologicalOrganizationWMOGuidelinesOn}. Climate change shifts local and global distributions of rainfall \cite[1605]{2021SeneviratneWeatherAnd}. These impacts are likely to occure in the near future and the one affecting airports are local in nature, e.g. flooding due to failures in drainage systems, but they may propagate throughout the aviation network depending on their severity \cite[28-34]{2018ICAOClimateAdaptation}.
