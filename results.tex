\section{Torino-Caselle airport}
The methodology of \gls{CCRA} is applied in particular to Torino-Caselle airport, located near the municipality of Caselle Torinese, in the Metropolitan City of Turin, Italy.
The central coordinates for the grid cells are $(\qty{45.203}{\degreeNorth}, \qty{7.649}{\degreeEast})$ in decimal degrees. These data are derived from \gls{ENAC}, the Italian civil aviation authority \cite{2014ENACTorinoCaselle}.

The construction of the airport in its modern settings started in 1949 and major upgrades to infrastructures happened before 2006 (see \cite[18]{2015PudduCorporateSocial} for a summary on the history of the airport). Therefore, the reference period is
\begin{equation}
  \label{eq:period_reference_torino}
  \prd{S_\text{time}}{ref} = \{ t : \text{$t$ day from \DTMdisplaydate{1971}{1}{1}{-1} to \DTMdisplaydate{2000}{12}{31}{-1}} \}
  \quad .
\end{equation}
For the present \gls{CCRA}, the airport is considered unaffected by structural changes starting from year 2006. Hence \glspl{indicator} values for that year or closest to it are chosen.

Values of \gls{exposure} indicators are shown in table~\ref{tab:drivers_exposure} and are gathered from \cite[154-179]{2010OneWorksAtlanteDegli}.

Same source is used for values of \gls{sensitivity}. A note on the {Age buildings} indicator is due: following \cite[5]{2022DeVivoRiskAssessment}, it is the difference in years between the time of writing and the start of the construction of the airport, since the oldest components are supposed to be the most vulnerable. Because Torino-Caselle airport was subject to various changes, dating back before its designation for civil aviation, the year 1949 is approximated as the starting year for all operations and the lifecycle of the airport components. Instead, structural changes are supposed to end in year 2006, hence the age is evaluated between years 1949 and 2006. This is an heavy approximation because {Age buildings} is an indicator which directly states the passing of time. However, for the purposes of this study, this level of accuracy is considered acceptable as the focus is not on obtaining precise \gls{CCRA} results but on assessing how changes in parameters affect those results.
A more dynamic \gls{CCRA} methodology may be an interesting extension of the present study.

\Glspl{indicator} of \gls{adaptive_capacity} are mainly categorical and they are gathered from \cite[66-79]{2015PudduCorporateSocial}. For use in calculations, categorical values are converted to a numerical scale following \cite[6]{2023DeVivoClimate-RiskAssessment}. The target values are in the interval $[0, 1]$, with lower values indicating lower \gls{risk}. More in detail, the presence of mitigation or adaptation strategies scores \num{0.1}, while their absence scores \num{0.9}. Intermediate values correspond to the existence of plans that have not yet been implemented. For this reason, the values for the indicators {Risk awareness} and {Guidelines for adaptation plan to climate change} are set considering the planning for adaptation and mitigation strategies from the years after 2006.
